\usepackage[pdftex,a4paper]{geometry}
\usepackage{pdfpages}
\usepackage{graphicx}
\usepackage{amssymb}
\usepackage{gensymb}
\usepackage{eurosym}
\usepackage{ifthen}
\usepackage{float}
\usepackage{calc}
\usepackage[export]{adjustbox}% http://ctan.org/pkg/adjustbox
\usepackage{xstring}
\usepackage{multicol}
\usepackage[framemethod=TikZ]{mdframed}
\usepackage[T1]{fontenc}

% Switch font comprehensively
\usepackage[scaled]{helvet}
\renewcommand\familydefault{\sfdefault} 

% Use as much of the page as we can.
\pagestyle{empty}
\setlength{\parindent}{4em}
\setlength{\parskip}{1.2em}
\setlength{\textheight}{28cm}
\setlength{\textwidth}{19cm}
\setlength{\hoffset}{-2cm}
\setlength{\voffset}{-2cm}

% DECAL function
%
% The decal function takes 5 inputs and extracts a symbol from the Italian ISO document that is included in this repository. The function then displays the symbol together with the title and warning text.
%
% Parameters:		#1 Page with the symbol on it in the document.
%			#2 Title
%			#3 Y height on the page.
%			#4 X location on the page. (not sure on this)
%			#5 Warning subtext
%
\newcommand{\decal}[5]{
\begin{minipage}{\linewidth}
	\begin{minipage}[t]{0.3\textwidth}
	       \vspace{0pt}
	       % The ISO document has a 'gutter' - so we shift the X a bit on the left/right page.
	       %
		\ifthenelse{\isodd{#1}}{ %if the pagenumber is odd, use this location for the image
			\includegraphics[width=0.9\textwidth,page=#1,viewport=103 #3 291 #4,clip=true]{13GR_PistopioimeniSimansi_ISO_7010.pdf}
		}{ %else (because the pagenumber is even), use this location for the image.
			\includegraphics[width=0.9\textwidth,page=#1,viewport=70 #3 261 #4,clip=true]{13GR_PistopioimeniSimansi_ISO_7010.pdf}
		}
	\end{minipage}
	\begin{minipage}[t]{0.6\textwidth}
		\setlength{\parindent}{4em}
		\setlength{\parskip}{1.2em}
	       \vspace{0pt}\raggedright
		{\LARGE \textbf{#2}} %include title text in bold and such
	
		#5 %include warning subtext
		\vspace{0.5cm}
	\end{minipage}
\end{minipage}
}


% Some home made symbols which are not part of the official ISO list of symbols:
%
% Two person symbol
\newcommand{\twopersons}{
\raisebox{-.5\height}{\includegraphics[width=6mm]{File:Aiga_toiletsq_men}}\textbf{x 2} 
}

% Some home made symbols which are not part of the official ISO list of symbols:
% Clock with 7-19 symbol
%
\newcommand{\seventoseven}{
\raisebox{-.5\height}{\includegraphics[width=10mm]{clock-limits}}
}


% Define classes for Action, Prohibit and Warn. These each take three inputs.
%
% Turns out that the symbols extracted from the ISO document with \decal are a bit 
% shifted in the Y position or size depending on what class they are. So we split
% them into the three sections that have height differences.
%
% Parameters:		Page with the symbol on it. Actual pagenumber, not the number on the page!
%			Title
%			Warning subtext
%
\newcommand{\action}[3]{\decal{#1}{#2}{520}{715}{#3}}
\newcommand{\prohib}[3]{\decal{#1}{#2}{495}{690}{#3}}
\newcommand{\warn}[3]{\decal{#1}{#2}{520}{715}{#3}}

%% Command to create a round box to call extra attention to something.
%%
\newcommand{\smallicontext}[2]{%
\begin{mdframed}[roundcorner=4pt] 
\begin{tabular}{lp{0.8\linewidth}}
#1 & #2 \\
\end{tabular}
\end{mdframed}
}

% machinePage
% A full page with a machine on it. Takes 5 inputs.
%
% Params:		#1 Machine name
%			#2 Tokens for standard instrutions: Zero or more from
%				Approval	Trustee approval needed.
%				Noise		Not after 19:00
%				Two		Two people required
%				Log		use must be logged
%				Dewalt		Attach dewalt dutch extractor.
%			#3 Specific instructions - added to the above text.
%			#4 Warnings, Symbols - main symbols on page.
%			#5 Extra footer/bottom of the page text
%
\newcommand{\machinePage}[5]{%
\begin{center}
	\vspace{0cm}
	{\fontsize{50}{60} \textbf{#1}}%Machine name
 
	\action{49}{% Determines if you require Induction / the tool has mandatory instructions / or general safety instructions
		\textbf{%
 		\IfSubStr{#2}{Induction}%
			{An induction is Mandatory}
			{\IfSubStr{#2}{NoMandatory}{Safety Rules}{These Instructions are Mandatory}}%
		}}
		{%
		\IfSubStr{#2}{NoMandatory}{}{Read these instructions \textbf{prior} to use to prevent damage to you and the equipment. Check the wiki for more information.}
		\IfSubStr{#2}{Induction}{You must have completed an induction to use the machine.}{}
			
		#3 % added to above instructions
		
		\IfSubStr{#2}{Noise}{\smallicontext{\seventoseven}{
			Machine can only be used between 07:00 - 19:00 (be kind to our neighbours) %not used in Bristol
			}
		}{}

		\IfSubStr{#2}{Two}{\smallicontext{\twopersons}{A second person must be present during operation, have agreed to be your second, knows how to stop the machine and what to do in an emergency, including opening the door for paramedics if needed.	} %not used in Bristol
		}{}
		\IfSubStr{#2}{Online}{Induction can be performed online. Complete the Induction to receive an unlock code. Make sure you lock up the machine after use!}{}
		\IfSubStr{#2}{InPerson}{Induction in person is required. Post on the forum to arrange this.}{}

		\IfSubStr{#2}{Log}{Report any use in the log - and \textbf{pay for it}!}{}

		\IfSubStr{#2}{Dust}{Be sure to attach and power on the dust extractor. Move the gates to extract dust from your machine.}{}
	}
	\begin{multicols}{2}#4\end{multicols}
	
	\vspace{0.2cm}
	#5
%footer text at the bottom of the page.

	If a machine is not clean or appears broken - then report this to the forum. Clean/fix prior to use.

	\textbf{Report any damage, issues or accident within 24 hours to the forum.}

        Notice something wrong with this poster? Edit directly at https://github.com/bristolhackspace/SafetySheetsMachines/ or request a change in the forum.
\end{center}
\vfill
\begin{flushright}
{\tiny \version -- \today}
\end{flushright}
\pagebreak
}





